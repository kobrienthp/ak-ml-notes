%%%%%%%%%%%%%%%%%%%%%%%%%%%%%%%%%%%%%%%%%%%%%%%%%%%%%%%%%%%%%%%%%%%%%%%%%%%%%%%%%%%%%%
%
% File:			headerPhysics.tex
%
% Description:	Contains (mostly) useful commands for writing mathematics and physics.
%
%%%%%%%%%%%%%%%%%%%%%%%%%%%%%%%%%%%%%%%%%%%%%%%%%%%%%%%%%%%%%%%%%%%%%%%%%%%%%%%%%%%%%%

\makeatletter % Need for anything that contains an @ command 
\renewcommand{\labelenumi}{(\alph{enumi})} % Use letters for enumerate
% \DeclareMathOperator{\Sample}{Sample}
\let\vaccent=\v % rename builtin command \v{} to \vaccent{}


\newcommand{\expv}[1]{\langle #1 \rangle}
\newcommand{\vareps}{\varepsilon}
\newcommand{\tbf}[1]{\textbf{#1}}
\newcommand{\bv}[1]{\bm{#1}} % for vectors

\renewcommand{\Re}{\mathfrak{Re}}
\renewcommand{\Im}{\mathfrak{Im}}

\newcommand{\prm}{\ensuremath{^{\prime}}}

\newcommand{\op}[1]{\ensuremath{\hat{#1}}}

\renewcommand{\i}{\ensuremath{\text{i}}} % imaginary

\newcommand{\donehalf}{\dfrac{1}{2}}
\newcommand{\onehalf}{\frac{1}{2}}

\newcommand{\infint}[1]{\int\limits_{-\infty}^{\infty}{#1}}
\newcommand{\trace}{\text{Tr}}

\newcommand{\meas}[2]{\text{d}^\ensuremath{{#1}\text{#2}}\,}
%\newcommand{\m}[2]{\ensuremath{\text{#1 #2}}}

\newcommand{\gv}[1]{\ensuremath{\mbox{\boldmath$ #1 $}}} 
% for vectors of Greek letters

\newcommand{\uv}[1]{\ensuremath{\mathbf{\hat{#1}}}} % for unit vector
\newcommand{\abs}[1]{\left| #1 \right|} % for absolute value
\newcommand{\Avg}[1]{\langle #1 \right\rangle} % for average
\newcommand{\avg}[1]{\langle #1 \rangle} % for average
\let\underdot=\d % rename builtin command \d{} to \underdot{}
\renewcommand{\d}[2]{\frac{d #1}{d #2}} % for derivatives
\newcommand{\dd}[2]{\frac{d^2 #1}{d #2^2}} % for double derivatives
\newcommand{\pd}[2]{\frac{\partial #1}{\partial #2}} 
% for partial derivatives
\newcommand{\pdd}[2]{\frac{\partial^2 #1}{\partial #2^2}} 
% for double partial derivatives

\newcommand{\pdc}[3]{\left( \frac{\partial #1}{\partial #2}
 \right)_{#3}} % for thermodynamic partial derivatives
\newcommand{\dfr}[2]{\dfrac{#1}{#2}}

\newcommand{\ketbra}[1]{\left| #1\rangle \langle #1\right|}
\newcommand{\matrixel}[3]{\left< #1 \vphantom{#2#3} \right|
 #2 \left| #3 \vphantom{#1#2} \right>} % for Dirac matrix elements
\newcommand{\grad}[1]{\gv{\nabla} #1} % for gradient
\let\divsymb=\div % rename builtin command \div to \divsymb
\renewcommand{\div}[1]{\gv{\nabla} \cdot #1} % for divergence
\newcommand{\curl}[1]{\gv{\nabla} \times #1} % for curl
\let\accent=\= % rename builtin command \= to \baraccent
\renewcommand{\=}[1]{\stackrel{#1}{=}} % for putting numbers above =
% \newtheorem{prop}{Proposition}
% \newtheorem{thm}{Theorem}[section]
% \newtheorem{lem}[thm]{Lemma}
% \theoremstyle{definition}
% \newtheorem{dfn}{Definition}
% \theoremstyle{remark}
% \newtheorem*{rmk}{Remark}

% Bra-Ket notation
\newcommand{\Ket}[1]{\left| #1 \right\rangle} % for Dirac kets
\newcommand{\Bra}[1]{\left\langle #1 \right|} % for Dirac bras 
\newcommand{\ket}[1]{| #1 \rangle} % for Dirac kets
\newcommand{\bra}[1]{\langle #1 |} % for Dirac bras 
\newcommand{\Braket}[2]{\left\langle #1 \vphantom{#2} \right|
 \left. #2 \vphantom{#1} \rangle>} % for Dirac brackets

% (Anti-) Commutators
\newcommand{\comm}[2]{\ensuremath{\left[#1,~#2\right]}}
\newcommand{\anticomm}[2]{\ensuremath{\left\{#1,~#2\right\}}}

\newcommand{\mc}[1]{\ensuremath{\mathcal{#1}}}
\newcommand{\mb}[1]{\ensurement{\mathbb{#1}}}

% (Dirac)-Adjoint
\renewcommand{\dag}{\ensuremath{^{\dagger}}}
\newcommand{\dadj}[1]{\ensuremath{\overline{#1}}}
\newcommand{\ansatz}[1]{\ensuremath{\overline{#1}}}
\newcommand{\cc}[1]{\ensuremath{\overline{#1}}}

% Identity operator
%\newcommand{\id}{\ensuremath{\mathds{1}}}
\newcommand{\id}{\ensuremath{\mathbbm{1}}}
\newcommand{\ZZ}{\ensuremath{\mathbb{Z}_{2}}}
\newcommand{\NN}{\ensuremath{\mathbb{N}}}
\newcommand{\RR}{\ensuremath{\mathbb{R}}}

% Inverse operation
\newcommand{\inv}{\ensuremath{^{-1}}}

% Tilde
\newcommand{\til}[1]{\ensuremath{\widetilde{#1}}}

% Lithium Iridates
\newcommand{\sodiumIridate}{Na$_2$IrO$_3$}
\newcommand{\alphaLithiumIridate}{$\alpha$-Li$_2$IrO$_3$}
\newcommand{\betaLithiumIridate}{$\beta$-Li$_2$IrO$_3$}
\newcommand{\gammaLithiumIridate}{$\gamma$-Li$_2$IrO$_3$}
\newcommand{\ruthiniumTriChloride}{RuCl$_3$}

% Effective Hamiltonian
\newcommand{\heff}{\ensuremath{\til{H}_\text{eff}}}

% Weyl Hamiltonian
\newcommand{\hweyl}{\ensuremath{H_\text{Weyl}}}

% Often used vectors
\newcommand{\be}{\bv{e}}
\newcommand{\bR}{\bv{R}}
\newcommand{\br}{\bv{r}}
\newcommand{\bz}{\bv{z}}
\newcommand{\bx}{\bv{x}}
\newcommand{\by}{\bv{y}}
\newcommand{\ba}{\bv{a}}
\newcommand{\bq}{\bv{q}}
\newcommand{\bk}{\bv{k}}
\newcommand{\bp}{\bv{p}}
\newcommand{\bP}{\bv{P}}
\newcommand{\bb}{\bv{b}}
\newcommand{\bc}{\bv{c}}
\newcommand{\bsigma}{\bm{\sigma}}
\newcommand{\btau}{\bm{\tau}}
\newcommand{\bS}{\bv{S}}

% Signum
\DeclareMathOperator{\signum}{sgn}

% Latin abbreviations
\newcommand{\ie}{{\it i.e.}}
\newcommand{\eg}{{\it e.g.}}

\DeclareMathOperator*{\argmin}{arg\,min}
\DeclareMathOperator*{\softmax}{\text{\textit{softmax}}}

\newlength{\Lnote}
\newcommand{\notte}[1]{
    \noindent\rule{\dimexpr\textwidth}{1pt}\\
    \mbox{}\hspace{-\Lnote}\textbf{Note:~}
    #1\\[-0.5ex] 
   \rule{\dimexpr\textwidth}{1pt}
}